\chapter{Wahrscheinlichkeitsverteilungen}

\begin{definition}{Wahrscheinlichkeitsverteilung}{verteilung}
Sei $A$ eine \link{def:ealg}{Ereignisalgebra}, $X$ eine \link{def:zvar}{Zufallsvariable}
mit Zustandsraum $S$. Dann heißt Funktion $P: S \rightarrow [0,1]$ definiert
durch
\[P(A) = P(X^{-1}(A)), A \in S\]

eine \defw{Wahrscheinlichkeitsverteilung}. Eine Verteilung einer stetigen
Zufallsvariable wird als \defw{stetige Verteilung}, die einer diskreten
Zufallsvariable als \defw{diskrete Verteilung} bezeichnet.
\end{definition}

Hinweis: Diese Definition beschreibt eigentlich die Verteilung einer
Zufallsvariablen. Da in der Regel aber nicht der Grundraum eines
Zufallsexperiments, sondern eine Zusammenfassung (z.B. in Form einer
Zufallsvariablen) betrachtet wird, kann man sich etwas Definitionsaufwand
sparen.

\section{Diskrete Verteilungen}

Im Folgenden Abschnitt sei $X$ eine diskrete \link{def:zvar}{Zufallsvariable}
mit Zustandsraum $S$, $A \in S$ und $P$ eine
\link{def:verteilung}{Wahrscheinlichkeitsverteilung} dieser Zufallsvariable.


\subsection{Gleichverteilung}

Die Gleichverteilung ist eine sehr einfache Verteilung, bei der jeder Wert der
Zufallsvariable mit der gleichen Wahrscheinlichkeit auftritt:
\[P(A) = \frac{|A|}{|S|}\]


\subsection{Bernoulli-Verteilung ($B(p)$)}

Die Bernoulli-Verteilung beschreibt eine Zufallsvariable, die nur zwei mögliche
Zustände besitzt, hier bezeichnet als $S = \{0,1\}$. Der beliebig, aber fest
gewählte Ausgang $A$ besitzt die Wahrscheinlichkeit $0 \le p \le 1$, sodass
gilt:
\[P(X=1)=p\]
\[P(X=0)=1-p\]


\subsection{Binomialverteilung ($B(n,p)$)}

Die Binomialverteilung beschreibt die $n$-fache Durchführung eines Experiments
mit nur zwei komplementären Ausgängen werden. Die Zufallsvariable $X$ gibt an,
wie oft bei $n$-facher Wiederholung der beliebig, aber fest gewählte Ausgang $A$
eintritt. Damit kann $X$ die Werte $0, ..., n$ annehmen. Die Wahrscheinlichkeit
$p$ des Eintretens des gewählten Ausgangs bleibt dabei über alle $n$
Wiederholunge gleich. Es gilt:
\[P(X=k) = \binom{n}{k}p^k(1-p)^\{n-k\}\]


\subsection{Geometrische Verteilung ($Geo(p)$)}

Eine geometrische Verteilung entsteht durch die Wiederholung eines
Wahrscheinlichkeitsexperiments mit zwei komplementären Ausgängen. Die
Zufallsvariable $X$ beschreibt die Anzahl an Versuchen, die durchgeführt werden
müssen, bis der beliebig, aber fest gewählte Ausgang $B$ eintritt.

Der Zustandsraum von $X$ ist damit $\N_0$. Sei $p$ die Wahrscheinlichkeit, dass
Ausgang $B$ eintritt. Die Wahrscheinlichkeit, dass nach $k$ Wiederholungen der
Ausgang $B$ das erste mal auftritt, ist:
\[P(X=k) = (1-p)^kp\]


\subsection{Poisson-Verteilung ($Poi(\lambda)$)}

Die Poisson-Verteilung entsteht bei Vorgängen, die im Durchschnitt mit
konstanter Rate $\lambda \in (0, \infty)$ in einem beliebigen, aber
festen Zeitintervall auftreten. Die Zufallsvariable $X$ beschreibt, wie viele
Vorgänge tatächlich in dem Zeitintervall aufgetreten sind, sodass $S=\N_0$ gilt.

Die Wahrscheinlichkeit, dass in dem Zeitintervall genau $k$ Vorgänge
stattfinden, beträgt:
\[P(X=k) = \frac{\lambda^k}{k!}\,\e^{-\lambda}\]


\section{Stetige Verteilungen}

\begin{definition}{Wahrscheinlichkeitsdichte}{dichte}
Sei $X$ eine stetige \link{def:zvar}{Zufallsvariable} mit Zustandsraum $S$.
Eine Funktion $\rho: \R \rightarrow \R$ heißt \defw{Wahrscheinlichkeitsdichte},
wenn gilt:
\begin{align*}
  \forall x: \rho(x) \ge 0 \\
  \int \rho(x) \,\mathrm{d}x = 1
\end{align*}
\end{definition}

Die Wahrscheinlichkeitsdichte kann verwendet werden, um die Wahrscheinlichkeit,
dass X bestimmte Werte annimmt, zu berechnen ($a,b\in\R$):
\begin{align*}
  P(X < b) = \int_{-\infty}^{b}\rho(x)\,\mathrm{d}x\\
  P(a<X<b) = \int_{a}^{b}\rho(x)\,\mathrm{d}x\\
  P(X < b) = \int^{\infty}_{b}\rho(x)\,\mathrm{d}x
\end{align*}

Im Folgenden Abschnitt sei $X$ eine stetige \link{def:zvar}{Zufallsvariable}
mit Zustandsraum $S$, $A \in S$ und $P$ eine
\link{def:verteilung}{Wahrscheinlichkeitsverteilung} dieser Zufallsvariable.


\subsection{Gleichverteilung ($U(a,b)$)}

Die gleichverteilte Zufallsvariable $X$ nimmt die Werte $S=(a,b)$ an. Für die
Wahrscheinlichkeitsdichte gilt:
\[\rho(x) = \frac{1}{b-a}\cdot\mathbb{I}_{(a,b)}(x)\]

Dabei bezeichnet $\mathbb{I}_{(a,b)}$ die \defw{Indikatorfunktion}, die Werte im
Intervall von $(a,b)$ auf $1$ und alle anderen Werte auf $0$ abbildet.


\subsection{Exponentialverteilung ($Exp(\lambda)$)}

Für eine Exponentialverteilung mit konstanter Ereignisrate $\lambda>0$ gilt:
\[\rho(x) = \lambda \e^{-\lambda x}\cdot\mathbb{I}_{(0, \infty)}\]

\subsection{Normalverteilung ($N(\mu, \sigma^2)$)}

In der Natur kommen Normalverteilungen vor wenn sich eine große Anzahl
unabhängiger Verteilungen überlagern. Für die Wahrscheinlichkeitsdichte gilt:
\[\rho(x) = \frac{1}{\sqrt{2\pi\sigma^2}}\,exp(-\frac{(x-\mu)^2}{2\sigma^2})\]


\subsection{Standardnormalverteilung ($N(0,1)$)}

Eine standardnormalverteilte Zufallsvariable nimmt im Mittel den Wert $0$ mit
einer Varianz von $1$ an. Es gilt:
\[\rho(x) = \frac{1}{\sqrt{2\pi}}exp(-\frac{x^2}{2})\]

Die Wahrscheinlichkeit
\[P(X \le z) = \int_{-\infty}^{z}\rho(x)\,\mathrm{d}x\]

wird auch mit $\Phi(z)$ bezeichnet.

Ist $X$ normalverteilt mit $\mu$ und $\sigma^2$, dann ist die Zufallsvariable
\[Z=\frac{X-\mu}{\sigma}\]

standardnormalverteilt. Es gilt:
\begin{align*}
P(X>a) &= 1 - \Phi\Big(\frac{a-\mu}{\sigma}\Big) \\
P(X\le b) &= \Phi\Big(\frac{b-\mu}{\sigma}\Big) \\
P(a < X < b) &= \Phi\Big(\frac{b-\mu}{\sigma}\Big) - \Phi\Big(\frac{a-\mu}{\sigma}\Big)
\end{align*}
