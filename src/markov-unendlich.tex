Bisher haben wir endliche Markovketten betrachtet. Markovketten können jedoch
auch einen unendlich großen Zustandsraum wie $S=\N$ besitzen (der Zustandsraum
muss jedoch abzählbar unendlich sein). Dann können wir wie zuvor auch
\begin{itemize}
\item die Übergangsmatrix $\Pi$ aufstellen,
\item Äquivalenzklassen betrachten,
\item Interaktionsgraphen untersuchen,
\item Zustände bezüglich ihrer Periode untersuchen.
\end{itemize}

Allerdings ist das Rückkehrverhalten und die stationäre Verteilung bei
ergodischen unendlichen Markovketten anders. Wir definieren daher die Begriffe
\link{def:mk-rekurr}{rekurrent} und \link{def:mk-trans}{transient} neu und
unterscheiden unendliche und endliche Rückkehrzeit bei rekurrenten Zuständen.

\begin{definition}{Rekurrent, null-rekurrent, transient}{mk-inf-rekurr}
Sei $(X)_{n\in N_0}$ eine Markovkette mit abzählbar unendlich großem
Zustandsraum $S$. Dann heißt ein Zustand $y\in S$ \defw{rekurrent}, falls die
erwartete Rückkehrzeit in den Zustand endlich ist.

$y$ heißt \defw{null-rekurrent}, falls die Rückkehrzeit unendlich, die Rückkehr
aber sicher ist, d.h mit Wahrscheinlichkeit 1 stattfindet.

Zustand $y$ heißt \defw{transient}, falls mit positiver Wahrscheinlihckeit keine
Rückkehr zu $y$ stattfindet.
\end{definition}

\warn{Hier ist nicht klar, ob diese Definitionen äquivalent zu den bisherigen
Definitionen sind, oder ausschließlich für unendliche Markovketten gelten.}
