\chapter{Markov-Ketten}

Markov-Ketten dienen zur Beschreibung von Prozessen, deren zukünftiger Zustand
nur durch den letzten Zustand bestimmt wird. Markov-Prozesse werden darum auch
als "`gedächtnislos"' bezeichnet.

\begin{definition}{Stochastischer Prozess}{stochp}
Sei $T=\N_0$ und $S \subset\R$. Für jedes $t\in T$ sei $X_t$ eine
\link{def:zvar}{Zufallsvariable} mit Zustandsraum $S$. Dann heißt die Familie
\[
\big(X_t\big)_{t\in T}
\]
\defw{stochastischer Prozess} mit Zustandsraum $S$ und diskreter Zeit. Ist
$T = [0, \infty)$ handelt es sich um einen \defw{stetigen stochastischen Prozess}.
\end{definition}

\begin{definition}{Markov-Kette}{mk}
Ein stochastischer Prozess $\big(X_t\big)_{t\in \N_0}$ mit Zustandsraum $S$
heißt \defw{Markov-Kette} falls für alle $n\in\N_0$ und $k,l,x_0,x_1,...,x_n \in S$
gilt:
\[
P\big(X_{n+1} = l | X_{n}=k, X_{n-1}=x_{n-1},...,X_0=x_0\big) =
P\big(X_{n+1}=l|X_n=k\big)
\]
Diese Wahrscheinlichkeit heißt Übergangswahrscheinlichkeit und wird mit $p(k,l)$
bezeichnet.
\end{definition}

Die Definition der Markov-Kette beschreibt genau die Eigenschaft der
"`Gedächtnislosigkeit"': Die Wahrscheinlichkeit, in den nächsten Zustand zu
wechseln hängt lediglich davon ab, wo man sich gerade befindet (und nicht von
den Schritten davor).

\begin{definition}{Übergangsmatrix einer Markovkette}{mk-ümatr}
Sei $(X)_{n\in N_0}$ eine \link{def:mk}{Markovkette} mit Zustandsraum $S$. Die
Übergangswahrscheinlichkeiten $p(i,j)$ mit $i,j\in S$ lassen sich als Matrix
anordnen:
\[
\Pi = \big(p(i,j)\big)_{i,j\in S}
\]
Diese Matrix wird als \defw{Übergangsmatrix} der Markov-Kette bezeichnet.
\end{definition}

Jede Zeile der Übergangsmatrix enthält die Wahrscheinlichkeiten, in den Zustand
der jeweiligen Spalte überzugehen. Die Matrix ist eine sogenannte
\defw{stochastische Matrix}, das heißt alle Einträge besitzen Werte zwischen $0$
und $1$ und die Summe jeder Zeile ist $1$.

\begin{definition}{Verteilung einer Markovkette}{mk-vert}
Sei $(X)_{n\in N_0}$ eine \link{def:mk}{Markovkette} mit Zustandsraum $S$. Dann
heißt der Zeilenvektor mit $m\in \N_0$
\[
\pi_m = \big(P(X_m=s_1), ...,P(X_m=s_n)\big),\ s_1, ..., s_n \in S
\]
Verteilung der Markovkette zur Zeit $m$.
\end{definition}

\begin{theorem}{Berechnung einer Markovkette}{mk-ber}
Sei $S$ eine diskrete Menge, $\Pi = \big(p(k,l)\big)_{k,l\in S}$ eine
stochastische Matrix auf $S$ und $\pi_0 = \big(p_0(k)\big)$ die
\link{def:mk-vert}{Verteilung} der Zustände zu Beginn der Betrachtung. Dann ist
die Verteilung $\pi_n$ nach $n$ Schritten berechenbar durch
\[
\pi_n = \pi_0\cdot\Pi^n
\]
Die Matrix $\Pi^n$ gibt die Wahrscheinlichkeit an, in $n$ Schritten von
Zustand $i$ in Zustand $j$ überzugehen.
\end{theorem}

Damit hängt die Wahrscheinlichkeit, sich nach einer festen Anzahl an Schritten
in einem bestimmten Zustand zu befinden, neben den Übergangswahrscheinlichkeiten
$\Pi$ nur von der Anfangsverteilung ab.

\begin{theorem}{Chapman-Kolmogorow-Gleichung}{chapman}
Sei $(X)_{n\in N_0}$ eine Markovkette mit Zustandsraum $S$. Dann kann die
Wahrscheinlichkeit, in $n+m$ Schritten von Zustand $i$ in Zustand $k$ zu
wechseln gleich der Summe der Pfade über alle möglichen Zwischenstationen:
\[
P(X_{n+m}=k|X_0=i) = \sum_{j\in S}P(X_{n+m}=k|X_n=j)\cdot P(X_n=j|X_0=i)
\]
\end{theorem}

Das lässt sich durch die Definition der Matrixmultiplikation zeigen
(\href{https://de.wikipedia.org/wiki/Chapman-Kolmogorow-Gleichung}{Wikipedia}).

\section{Eigenschaften}

\begin{definition}{Pfad}{mk-pfad}
Ein konkreter Folge von Zuständen einer Markovkette $(X)_{n\in N_0}$ wird
als \defw{Pfad} bezeichnet.
\end{definition}

\subsection{Eigenschaften von Zuständen}

Die Zustände von Markovketten können zum Beispiel nach ihrer Erreichbarkeit
untereinander unterschieden werden.

\begin{definition}{Interaktionsgraph}{mk-igraph}
Sei $(X)_{n\in N_0}$ eine Markovkette mit Zustandsraum $S$. Der gerichtete Graph
$G=(V,E)$ mit Kanten $V=S$ und $E = \{(x,y) \in S\times S\ |\ p(x,y) \ne 0\}$ wird
als \defw{Interaktionsgraph} bezeichnet.
\end{definition}

Der Interaktionsgraph beschreibt also die direkten
Verbindungen der Zustände im Zustandsraum.

\begin{definition}{Erreichbarkeit}{mk-erreichbark}
Sei $(X)_{n\in N_0}$ eine Markovkette mit Zustandsraum $S$. Ein Zustand $y\in S$
heißt erreichbar von $x\in S$, falls es ein $n>0$ gibt, sodass gilt:
\[
P(X_n=y|X_0=x) > 0
\]
Erreichbarkeit kann auch durch die \link{def:ümatr}{Übergangsmatrix} $\Pi$
definiert werden:
\[
\exists n\in\N: \Pi^n(x,y) \ne 0
\]
\end{definition}

Ist $y$ von $x$ erreichbar, existiert im Interaktionsgraph ein Pfad von $x$ zu
$y$. Dieser Pfad muss nicht direkt sein, sondern kann auch über andere Zustände
führen. Wir verwenden für diese Erreichbarkeit die Schreibweise $x\to y$.

\begin{definition}{Verbundene Zustände}{mk-verb}
Sei $S$ der Zustandsraum einer Markovkette. Die Zustände $x,y\in S$ heißen
\defw{verbunden}, wenn gilt:
\[
x\to y \land y \to x
\]
Verbundene Zustände werden durch das Zeichen $\lr$ gekennzeichnet
($x \lr y$).
\end{definition}

Die Verbundenheit von Zuständen ist eine Relation, die \emph{reflexiv} (jeder
Zustand ist mit sich selbst verbunden), \emph{symmetrisch} (verbundene
Zustände sind auch in der "`Gegenrichtung"' verbunden) und \emph{transitiv}
(sind $a$ mit $b$ und $b$ mit $c$ verbunden, ist auch $a$ mit $c$
verbunden). Damit ist die Relation $\lr$ eine Äquivalenzrelation\more{mfnf-ä-rel}.
Das bedeutet insbesondere, dass die Zustandsmenge $S$ durch die Verbundenheitsrelation in
Äquivalenzklassen zerlegt wird, in denen jeder Zustand mit jedem anderen Zustand
verbunden ist.

Zwischen Äquivalenzklassen kann sich nicht beliebig bewegt werden; insbesondere
kann eine einmal verlassene Klasse nicht wieder erreicht werden. (Gäbe es einen
"`Weg zurück"', wären auch jedes Paar von Zuständen aus beiden Klassen
miteinander verbunden. Das ist ein ein Widerspruch, denn dann müssten diese
Zustände ja in \emph{einer} gemeinsamen Äquivalenzklasse liegen.)

Die \link{def:mk-erreichbark}{Erreichbarkeit von Zuständen} können wir auf die
Äquivalenzklassen übertragen:
\begin{definition}{Ordnung von Zustandsklassen}{mk-ord}
Sei $S$ der Zustandsraum einer Markovkette, der durch die Verbundenheitsrelation
$\lr$ in Klassen $S_{/\lr}=\{G_1, G_2,...,G_n\}$ zerlegt wird. Auf diesen
Klassen definieren wir folgende Relation:
\[
\preceq\ =\{(G_i, G_j)\ |\ \exists s_i\in G_i, s_j\in G_j: s_i\to s_j\}\ \subseteq\ S_{/\lr}\times S_{/\lr}
\]
Die Relation bildet eine partielle Ordnung\more{mfnf-o-rel}, ist also reflexiv, anti-symmetrisch und transitiv. Für
$G_i\preceq G_j$ schreiben wir auch $G_i\to G_j$.
\end{definition}

Die so definierte Ordnung ist jedoch im Allgemeinen nicht \emph{total}, da nicht
für jedes Paar von Zustandsklassen definiert sein muss, ob $G_i\preceq G_j$ oder
$G_j \preceq G_i$ gilt. Das ist zum Beispiel der Fall, wenn die Klassen nicht
miteinander verbunden sind, also im Interaktionsgraphen unverbundene Subgraphen
bilden.

\begin{definition}{Abgeschlossene/Transiente Klassen}{mk-abg}
Sei $G$ eine Zustandsklasse einer Markovkette. $G$ heißt \defw{abgeschlossen},
wenn es keine Klasse gibt, die von $G$ erreicht werden kann:
\[
\nexists H\ne G: G\preceq H
\]
Ist $G$ nicht abgeschlossen, wird die Klasse als \defw{transient} bezeichnet.
\end{definition}

Abgeschlossene Klassen werden zusätzlich nach der Anzahl ihrer Zustände
unterschieden:

\begin{definition}{Rekurrente/absorbierende Klassen}{mk-rekkurr-abs}
Sei $G$ eine \link{def:mk-abg}{abgeschlossene Klasse}. Gilt $|G|=1$, heißt $G$
\defw{absorbierend}, sonst ($|G| > 1$) \defw{rekurrent}.
\end{definition}

Sind alle Zustände einer Markovkette miteinander verbunden, gibt es nur eine
Äquivalenzklasse:
\begin{definition}{Irreduzibilität}{mk-irred}
Sei $(X)_{n\in N_0}$ eine Markovkette mit Zustandsraum $S$, der durch die
Verbundenheitsrelation $\lr$ in die Menge $S_{/\lr}$ zerlegt wird. Gilt
\[
S_{/\lr}\ = \{S\}
\]
heißt die Markovkette \defw{irreduzibel}.
\end{definition}
\begin{lemma}
Die Klasse $S$ einer irreduziblen Markovkette ist abgeschlossen.
\end{lemma}
 
\subsection{Rückkehreigenschaften}

\section{Langzeitverhalten}
