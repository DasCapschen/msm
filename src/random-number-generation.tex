\chapter{Generierung von Zufallszahlen}

Dieses Kapitel behandelt Grundlagen und Methoden zur Generierung von beliebig
verteilten Zufallszahlen durch einen Zufallszahlengenerator mit gleichverteilten
Zufallszahlen.

\section{Zufallsvektoren}

\begin{definition}{Zufallsvektor}{zvektor}
Seien $X_1, ..., X_n$ \link{def:zvar}{Zufallsvariablen}. Die Zusammenfassung
\[\underline{X} = (X_1, ..., X_n)^T\]
zu einem Vektor heißt \defw{Zufallsvektor}. Sind alle Komponenten des Vektors
diskret beziehungsweise stetig, heißt der Zufallsvektor diskret beziehungsweise
stetig.
\end{definition}

Die Werte eines (zweidimensionalen) diskreten Zufallsvektors lassen sich in einer Matrix
zusammenfassen:

\begin{definition}{Gemeinsame Verteilung eines Zufallsvektors}{vert-zvektor}
Sei $\underline{X} = (X, Y)^T$ ein diskreter Zufallsvektor, wobei die
Zufallsvariable $X$ die Werte $x_0, x_1, ..., x_n$ und $Y$ die Werte $y_0, y_1, ...,
y_m$ annimmt. Dann bezeichnet die Matrix
\[(p_{ij})_{i=0,...,n;j=0,...,m} \qquad mit\ p_{ij} = P(X=x_i, Y=y_j)\]
die \defw{gemeinsame Verteilung} von $\underline{X}$.
\end{definition}

Die Summierung von Zeilen bzw. Spalten der Matrix werden als Randverteilung
bezeichnet:
\[p_{i.}:=P(X=x_i) = \sum_j p_{ij}\]
\[p_{.j}:=P(Y=y_j) = \sum_i p_{ij}\]
Für stetige Zufallsvariablen $X$ und $Y$ mit gemeinsamer \link{def:dichte}{Dichte}
$\rho_{(X, Y)}\ge 0$ können folgende \defw{Randdichten} abgeleitet werden:
\[\rho_X(x) = \int\rho_{(X,Y)}(x,y)\mathrm{d} y\quad x\in\R\]
\[\rho_Y(y) = \int\rho_{(X,Y)}(x,y)\mathrm{d} x\quad y\in\R\]

Analog zur \link{def:bedw}{bedingten Wahrscheinlichkeit} von Ereignissen lässt sich
auch für Zufallsvektoren eine bedingte Wahrscheinlichkeit definieren:

\begin{definition}{Bedingte Wahrscheinlichkeit/Bedingte Dichte}{bwahr-zvektor}
Sei $\underline{X} = (X, Y)^T$ ein diskreter Zufallsvektor mit
\link{def:vert-zvektor}{gemeinsamer Verteilung} $(p_{ij})_{i,j=0,1,...}$. Dann ist
mit
\[P(Y=y_j|X=x_i) := \frac{P(X=x_i, Y=y_j)}{P(X=x_i)} = \frac{p_{ij}}{p_{i.}}\]
die \defw{bedingte Wahrscheinlichkeit} von $Y=y_j$ unter Bedingung $X=x_i$ gegeben.

Ist $\underline{X}$ ein stetiger Zufallsvektor mit gemeinsamer Dichte
$\rho_{(X,Y)}$, so bezeichnen die Funktionen
\[\rho_{X|Y=y}(x) = \frac{\rho_{(X,Y)}(x,y)}{\rho_Y(y)}\]
\[\rho_{Y|X=x}(y) = \frac{\rho_{(X,Y)}(x,y)}{\rho_X(x)}\]
die \defw{bedingte Dichte} von $X$ unter $Y=y$ bzw. $Y$ unter $X=x$.
\end{definition}

Ebenso analog zur \link{def:bedw}{bedingten Wahrscheinlichkeit} von Ereignissen kann
der \link{satz:bayes}{Satz von Bayes} für diskrete bzw. stetige Zufallsvektoren
formuliert werden:
\[p_{ij} = P(Y=y_j|X=x_i)\cdot p_{i.}\]
\[\rho_{(X,Y)} = \rho_{Y|X=x}(y)\cdot\rho_X(x)\]
