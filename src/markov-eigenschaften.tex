Markov-Ketten dienen zur Beschreibung von Prozessen, deren zukünftiger Zustand
nur durch den letzten Zustand bestimmt wird. Markov-Prozesse werden darum auch
als "`gedächtnislos"' bezeichnet.

\begin{definition}{Stochastischer Prozess}{stochp}
Sei $T=\N_0$ und $S \subset\R$. Für jedes $t\in T$ sei $X_t$ eine
\link{def:zvar}{Zufallsvariable} mit Zustandsraum $S$. Dann heißt die Familie
\[
\big(X_t\big)_{t\in T}
\]
\defw{stochastischer Prozess} mit Zustandsraum $S$ und diskreter Zeit. Ist
$T = [0, \infty)$ handelt es sich um einen \defw{stetigen stochastischen Prozess}.
\end{definition}

\begin{definition}{Markov-Kette}{mk}
Ein stochastischer Prozess $\big(X_t\big)_{t\in \N_0}$ mit Zustandsraum $S$
heißt \defw{Markov-Kette} falls für alle $n\in\N_0$ und $k,l,x_0,x_1,...,x_n \in S$
gilt:
\[
P\big(X_{n+1} = l | X_{n}=k, X_{n-1}=x_{n-1},...,X_0=x_0\big) =
P\big(X_{n+1}=l|X_n=k\big)
\]
Diese Wahrscheinlichkeit heißt Übergangswahrscheinlichkeit und wird mit $p(k,l)$
bezeichnet.
\end{definition}

Die Definition der Markov-Kette beschreibt genau die Eigenschaft der
"`Gedächtnislosigkeit"': Die Wahrscheinlichkeit, in den nächsten Zustand zu
wechseln hängt lediglich davon ab, wo man sich gerade befindet (und nicht von
den Schritten davor).

\begin{definition}{Übergangsmatrix einer Markovkette}{mk-matr}
Sei $(X)_{n\in N_0}$ eine \link{def:mk}{Markovkette} mit Zustandsraum $S$. Die
Übergangswahrscheinlichkeiten $p(i,j)$ mit $i,j\in S$ lassen sich als Matrix
anordnen:
\[
\Pi = \big(p(i,j)\big)_{i,j\in S}
\]
Diese Matrix wird als \defw{Übergangsmatrix} der Markov-Kette bezeichnet.
\end{definition}

Jede Zeile der Übergangsmatrix enthält die Wahrscheinlichkeiten, in den Zustand
der jeweiligen Spalte überzugehen. Die Matrix ist eine sogenannte
\defw{stochastische Matrix}, das heißt alle Einträge besitzen Werte zwischen $0$
und $1$ und die Summe jeder Zeile ist $1$.

\begin{definition}{Verteilung einer Markovkette}{mk-vert}
Sei $(X)_{n\in N_0}$ eine \link{def:mk}{Markovkette} mit Zustandsraum $S$. Dann
heißt der Zeilenvektor mit $m\in \N_0$
\[
\pi_m = \big(P(X_m=s_1), ...,P(X_m=s_n)\big),\ s_1, ..., s_n \in S
\]
Verteilung der Markovkette zur Zeit $m$.
\end{definition}

Die Komponente einer Verteilung $\pi$ für einen Zustand $s$ wird mit $\pi(s)$
bezeichnet.

\begin{theorem}{Berechnung einer Markovkette}{mk-ber}
Sei $S$ eine diskrete Menge, $\Pi = \big(p(k,l)\big)_{k,l\in S}$ eine
stochastische Matrix auf $S$ und $\pi_0 = \big(p_0(k)\big)$ die
\link{def:mk-vert}{Verteilung} der Zustände zu Beginn der Betrachtung. Dann ist
die Verteilung $\pi_n$ nach $n$ Schritten berechenbar durch
\[
\pi_n = \pi_0\cdot\Pi^n
\]
Die Matrix $\Pi^n$ gibt die Wahrscheinlichkeit an, in $n$ Schritten von
Zustand $i$ in Zustand $j$ überzugehen.
\end{theorem}

Damit hängt die Wahrscheinlichkeit, sich nach einer festen Anzahl an Schritten
in einem bestimmten Zustand zu befinden, neben den Übergangswahrscheinlichkeiten
$\Pi$ nur von der Anfangsverteilung ab.

\begin{theorem}{Chapman-Kolmogorow-Gleichung}{chapman}
Sei $(X)_{n\in N_0}$ eine Markovkette mit Zustandsraum $S$. Dann kann die
Wahrscheinlichkeit, in $n+m$ Schritten von Zustand $i$ in Zustand $k$ zu
wechseln gleich der Summe der Pfade über alle möglichen Zwischenstationen:
\[
P(X_{n+m}=k|X_0=i) = \sum_{j\in S}P(X_{n+m}=k|X_n=j)\cdot P(X_n=j|X_0=i)
\]
\end{theorem}

Das lässt sich durch die Definition der Matrixmultiplikation zeigen
(\href{https://de.wikipedia.org/wiki/Chapman-Kolmogorow-Gleichung}{Wikipedia}).

\section{Eigenschaften}

\begin{definition}{Pfad}{mk-pfad}
Ein konkreter Folge von Zuständen einer Markovkette $(X)_{n\in N_0}$ wird
als \defw{Pfad} bezeichnet.
\end{definition}

\subsection{Zustandsklassen}

Die Zustände von Markovketten können zum Beispiel nach ihrer Erreichbarkeit
untereinander unterschieden werden.

\begin{definition}{Interaktionsgraph}{mk-igraph}
Sei $(X)_{n\in N_0}$ eine Markovkette mit Zustandsraum $S$. Der gerichtete Graph
$G=(V,E)$ mit Kanten $V=S$ und $E = \{(x,y) \in S\times S\ |\ p(x,y) \ne 0\}$ wird
als \defw{Interaktionsgraph} bezeichnet.
\end{definition}

Der Interaktionsgraph beschreibt also die direkten
Verbindungen der Zustände im Zustandsraum.

\begin{definition}{Erreichbarkeit}{mk-erreichbar}
Sei $(X)_{n\in N_0}$ eine Markovkette mit Zustandsraum $S$. Ein Zustand $y\in S$
heißt erreichbar von $x\in S$, falls es ein $n>0$ gibt, sodass gilt:
\[
P(X_n=y|X_0=x) > 0
\]
Erreichbarkeit kann auch durch die \link{def:mk-matr}{Übergangsmatrix} $\Pi$
definiert werden:
\[
\exists n\in\N: \Pi^n(x,y) \ne 0
\]
\end{definition}

Ist $y$ von $x$ erreichbar, existiert im Interaktionsgraph ein Pfad von $x$ zu
$y$. Dieser Pfad muss nicht direkt sein, sondern kann auch über andere Zustände
führen. Wir verwenden für diese Erreichbarkeit die Schreibweise $x\to y$.

\begin{definition}{Verbundene Zustände}{mk-verb}
Sei $S$ der Zustandsraum einer Markovkette. Die Zustände $x,y\in S$ heißen
\defw{verbunden}, wenn gilt:
\[
x\to y \land y \to x
\]
Verbundene Zustände werden durch das Zeichen $\lr$ gekennzeichnet
($x \lr y$).
\end{definition}

Die Verbundenheit von Zuständen ist eine Relation, die \emph{reflexiv} (jeder
Zustand ist mit sich selbst verbunden), \emph{symmetrisch} (verbundene
Zustände sind auch in der "`Gegenrichtung"' verbunden) und \emph{transitiv}
(sind $a$ mit $b$ und $b$ mit $c$ verbunden, ist auch $a$ mit $c$
verbunden). Damit ist die Relation $\lr$ eine Äquivalenzrelation\more{mfnf-ä-rel}.
Das bedeutet insbesondere, dass die Zustandsmenge $S$ durch die Verbundenheitsrelation in
Äquivalenzklassen zerlegt wird, in denen jeder Zustand mit jedem anderen Zustand
verbunden ist.

Zwischen Äquivalenzklassen kann sich nicht beliebig bewegt werden; insbesondere
kann eine einmal verlassene Klasse nicht wieder erreicht werden. (Gäbe es einen
"`Weg zurück"', wären auch jedes Paar von Zuständen aus beiden Klassen
miteinander verbunden. Das ist ein ein Widerspruch, denn dann müssten diese
Zustände ja in \emph{einer} gemeinsamen Äquivalenzklasse liegen.)

Um die Zustände einer Markovkette zu klassifizieren, beginnt man bei einem
beliebigen Zustand und bestimmt dessen transitive Hülle bezüglich $\lr$, also
alle Zustände die mit dem gewählten Zustand verbunden (nicht nur erreichbar!)
sind. Damit ist eine Klasse der Markovkette gefunden. Dieses Vorgehen
wird dann mit nicht klassifizierten Zuständen wiederholt, bis alle Zustände
klassifiziert sind.

\begin{example}{Bestimmung der Zustandsklassen}{}
Gesucht sind die Zustandsklassen eine Markovkette mit folgendem Interaktionsgraph:

\begin{centering}
\begin{tikzpicture}
  \node[v] (1) at (2,1) {1};
  \node[v] (2) at (4,2) {2};
  \node[v] (3) at (6,2) {3};
  \node[v] (6) at (5,0) {6};
  \node[v] (4) at (8,1) {4};
  \node[v] (5) at (10,1) {5};

  \draw[e] (1) to[loop above] ();
  \draw[e] (6) -- (1);
  \draw[e] (6) -- (3);
  \draw[e] (2) to[bend right] (3);
  \draw[e] (3) to[bend right] (2);
  \draw[e] (2) to[loop above] ();
  \draw[e] (2) -- (6);
  \draw[e] (6) -- (4);
  \draw[e] (4) to[bend left] (5);
  \draw[e] (5) to[bend left] (4);
\end{tikzpicture}
\end{centering}

Zustand 1 ist mit keinen weiteren Zuständen verbunden.
Zustand 2 ist mit Zuständen 3 und 6 verbunden, aber z.B. nicht mit Zustand 4
oder 1, da von dort aus kein Weg zurück führt.
Zustand 4 ist mit Zustand 6 verbunden.
Damit sind alle Zustände eingeordnet und die Zerlegung der Zustände ist:
\[
S_{/\lr} = \{\{1\}, \{2,3,6\}, \{4, 5\}\}
\]
\end{example}

Die \link{def:mk-erreichbar}{Erreichbarkeit von Zuständen} können wir auf die
Äquivalenzklassen übertragen:
\begin{definition}{Ordnung von Zustandsklassen}{mk-ord}
Sei $S$ der Zustandsraum einer Markovkette, der durch die Verbundenheitsrelation
$\lr$ in Klassen $S_{/\lr}=\{G_1, G_2,...,G_n\}$ zerlegt wird. Auf diesen
Klassen definieren wir folgende Relation:
\[
\preceq\ =\{(G_i, G_j)\ |\ \exists s_i\in G_i, s_j\in G_j: s_i\to s_j\}\ \subseteq\ S_{/\lr}\times S_{/\lr}
\]
Die Relation bildet eine partielle Ordnung\more{mfnf-o-rel}, ist also reflexiv, anti-symmetrisch und transitiv. Für
$G_i\preceq G_j$ schreiben wir auch $G_i\to G_j$.
\end{definition}

Die so definierte Ordnung ist jedoch im Allgemeinen nicht \emph{total}, da nicht
für jedes Paar von Zustandsklassen definiert sein muss, ob $G_i\preceq G_j$ oder
$G_j \preceq G_i$ gilt. Das ist zum Beispiel der Fall, wenn die Klassen nicht
miteinander verbunden sind, also im Interaktionsgraphen unverbundene Subgraphen
bilden oder wenn zwei unverbundene Klassen in einem dritten Zustand
"`zusammenlaufen"'.

\begin{definition}{Abgeschlossene Klasse}{mk-abg}
Sei $G$ eine Zustandsklasse einer Markovkette. $G$ heißt \defw{abgeschlossen},
wenn es keine Klasse gibt, die von $G$ erreicht werden kann:
\[
\nexists H\ne G: G\preceq H
\]
\end{definition}

Abgeschlossene Klassen werden zusätzlich nach der Anzahl ihrer Zustände
unterschieden:

\begin{definition}{Absorbierende Klasse/Markovkette}{mk-abs}
Sei $(X)_{n\in N_0}$ eine Markovkette. Ist $G$ eine \link{def:mk-abg}{abgeschlossene Klasse}
mit $|G|=1$, so heißt die Klasse $G$ \defw{absorbierend}.

Sind alle abgeschlossenen Klassen absorbierend, so heißt die
Markovkette \defw{absorbierend}.
\end{definition}

Sind alle Zustände einer Markovkette miteinander verbunden, gibt es nur eine
Äquivalenzklasse:
\begin{definition}{Irreduzibilität}{mk-irred}
Sei $(X)_{n\in N_0}$ eine Markovkette mit Zustandsraum $S$, der durch die
Verbundenheitsrelation $\lr$ in die Menge $S_{/\lr}$ zerlegt wird. Existiert nur
eine einzige Klasse, das heißt es gilt
\[
S_{/\lr}\ = \{S\}
\]
heißt die Markovkette \defw{irreduzibel}.
\end{definition}

Das ist genau der Fall, dass alle Zustände mit allen anderen Zuständen verbunden
sind, sodass genau eine Klasse besteht.

\begin{lemma}
Die Klasse $S$ einer irreduziblen Markovkette ist abgeschlossen.
\end{lemma}
 
\subsection{Rückkehreigenschaften}

In diesem Abschnitt werden Zustände von Markovketten bezüglich ihres
Rückkehrverhaltens untersucht\more{wiki-mk-properties}.

\begin{definition}{Rückkehrzeit}{mk-rueckk}
Sei $(X)_{n\in N_0}$ eine Markovkette mit Zustandsraum $S$ und $y\in S$
\link{def:mk-rekurr}{rekurrent} und der Startzustand $X_0$ der Markovkette.
Dann heißt
\[
T_y = min\{n>0:\ X_n = y\}
\]
die (zufällige) \defw{Rückkehrzeit} in den Zustand $y$. Zusätzlich bezeichnet
\[
M_y = \{k>0: P(T_y=k|X_0=y > 0\}
\]
die Menge der Zeitschritte, in denen ausgehend von $y$ wieder $y$ erreicht
werden kann.
\end{definition}

\begin{definition}{Periode eines Zustands}{mk-periode}
Sei $(X)_{n\in N_0}$ eine Markovkette mit Zustandsraum $S$ und $y\in S$ ein
Zustand, der \emph{nur} in $M_y=\{d, 2d, 3d, ...\}$ Schritten erreicht werden
kann, so heißt
\[
d = \mathrm{ggT}(M_y)
\]
die \defw{Periode} des Zustands $y$.
\end{definition}

\begin{definition}{Aperiodischer Zustand, aperiodische Markovkette}{mk-aperiod}
Sei $y$ ein Zustand einer Markovkette. Besitzt dieser Zustand die Periode 1, so
heißt $y$ \defw{aperiodisch}.

Sind alle Zustände einer Markovkette aperiodisch, heißt die Markovkette
\defw{aperiodisch}.
\end{definition}

\begin{definition}{Ergodische Markovkette}{mk-ergod}
Ist eine Markovkette mit endlich vielen Zuständen \link{def:mk-irred}{irreduzibel} und
\link{def:mk-aperiod}{aperiodisch}, so heißt sie \defw{ergodisch}.
\end{definition}

Die Zustände eines \link{def:stochp}{stochastischen Prozess} können danach
unterschieden werden, ob man in jedem Fall zu dem Zustand, in dem man gestartet
ist, zurückkehren kann (rekurrent) oder nicht (transient)
\more{brilliant-transience-recurrence}:

\begin{definition}{Rekurrenter Zustand}{mk-rekurr}
Sei $y$ ein Zustand einer Markovkette. Dann heißt $y$ \defw{rekurrent}, wenn
es für jeden von $y$ aus \link{def:mk-erreichbar}{erreichbaren} Zustand einen
Weg zurück zu $i$ gibt.
\end{definition}

\begin{definition}{Transienter Zustand}{mk-trans}
Sei $y$ ein Zustand einer Markovkette. Dann heißt $y$ \defw{transient}, wenn $y$
nicht rekurrent ist.
\end{definition}

Per Definition besteht bei transienten Zuständen die "`Gefahr"', nicht mehr in
diese zurückkehren zu können. Auf lange Sicht sind transiente Zustände daher nur
Zwischenzustände.

Die Eigenschaften \link{def:mk-rekurr}{Rekurrenz} und
\link{def:mk-trans}{Transienz} sind Klasseneigenschaften: Besitzt ein Zustand
eine dieser Eigenschaften, so auch alle anderen Zustände der Klasse. (Innerhalb
einer Klasse sind schließlich immer alle Zustände miteinander verbunden, es gibt
also immer einen Weg zurück zum Startzustand. Damit "`überträgt"' sich die
Transienz bzw. Rekurrenz eines Zustands auf die gesamte Klasse.)

\begin{lemma}
Jede \link{def:mk-abg}{abgeschlossene} Klasse ist rekurrent.

Begründung: Eine abgeschlossene Klasse kann nur betreten, aber nicht mehr
verlassen werden. Somit sind nur Zustände der Klasse erreichbar, sodass immer
zum Startzustand zurückgekehrt werden kann.
\end{lemma}

\begin{example}{Transiente/Rekurrente Zustandsklassen}{mk-tr-rek}
Gegeben sei eine Markovkette mit folgendem Interaktionsgraph:

\begin{tikzpicture}
  \node[v] (1) at (2,1) {1};
  \node[v] (2) at (4,2) {2};
  \node[v] (3) at (6,2) {3};
  \node[v] (6) at (5,0) {6};
  \node[v] (4) at (8,1) {4};
  \node[v] (5) at (10,1) {5};

  \draw[e] (1) to[loop above] ();
  \draw[e] (6) -- (1);
  \draw[e] (6) -- (3);
  \draw[e] (2) to[bend right] (3);
  \draw[e] (3) to[bend right] (2);
  \draw[e] (2) to[loop above] ();
  \draw[e] (2) -- (6);
  \draw[e] (6) -- (4);
  \draw[e] (4) to[bend left] (5);
  \draw[e] (5) to[bend left] (4);
\end{tikzpicture}

Die Markovkette besitzt die Zustandsklassen $S_{1}=\{1\}$, $S_{2,3,6} =\{2,3,6\}$
und $S_{4,5}=\{4, 5\}$. Zustand 1 ist rekurrent, da alle erreichbaren Zustände
(also nur der Zustand selbst) wieder einen Pfad zurück zum Zustand 1 besitzen.
Wenn man in 1 startet, kann es also nicht passieren, dass man nicht mehr dahin
zurückkommt.
Die Klasse $S_{2,3,6}$ ist transient, da man von dort aus auch in die Klassen
$S_1$ und $S_{4,5}$ kommen kann, von denen jedoch kein Pfad zurück in die
"`Startklasse"' führt. Die Klasse $S_{4,5}$ ist rekurrent, da sie (wie auch
$S_1$) nicht verlassen werden kann.
\end{example}
