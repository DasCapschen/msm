\chapter{Grundlagen der Wahrscheinlichkeitstheorie}

Die Wahrscheinlichkeitstheorie ist ein Teilgebiet der Mathematik, dass sich mit
der Formalisierung, Modellierung und Untersuchung von zufälligen Vorgängen
beschäftigt (\href{https://de.wikipedia.org/wiki/Wahrscheinlichkeitstheorie}
{Wikipedia}).

\section{Grundbegriffe}

\begin{definition}{Zufallsversuch}{zufallsversuch}
Ein Versuch oder Vorgang, der unter genau festgelegten Versuchsbedingungen
durchgeführt einen zufälligen Ausgang besitzt, wird als \defw{Zufallsversuch}
(auch: \altname{Zufallsexperiment} oder \altname{Wahrscheinlichkeitsexperiment})
bezeichnet.
\end{definition}


\begin{definition}{Grundraum}{grundraum}
Alle möglichen Ausgänge eines \link{def:zufallsversuch}{Zufallsversuchs}
bilden den \defw{Grundraum} $\Omega$ (auch: \altname{Ergebnismenge}). Die
Elemente des Grundraums werden als \altname{Elementarereignisse} bezeichnet.
\end{definition}


\begin{definition}{Ereignis}{ereignis}
Eine Teilmenge $A$ von $\Omega$ wird als Ereignis bezeichnet. Dabei bezeichnet
$A = \Omega$ das sichere Ereignis, dass immer eintritt und $A = \emptyset$ das
unmögliche Ereignis.
\end{definition}

\begin{example}{Würfeln mit einem einfachen Würfel}{ereignis}
Der Grundraum $\Omega = \{1,2,3,4,5,6\}$
sind die Augenzahlen des Würfels. Ein Ereignis $A_1 = \{1,3,5\}$ beschreibt das
eine ungerade Augenzahl, $A_2 = \{5,6\}$ das
eine Augenzahl $\ge 5$ gewürfelt wird.
\end{example}


\begin{definition}{Ereignisalgebra}{ealg}
Eine Menge von Ereignissen bezogen auf einen \link{def:grundraum}{Grundraum}
$\Omega$ bildet eine \defw{Ereignisalgebra} oder \altname{Ereignissystem}, wenn gilt:

  \begin{itemize}
    \item Das sichere und das unmögliche Ereignis sind teil der Ergebnisalgebra.
    \item{Für jedes Ereignis $A$ gibt es ein komplementäres Ereignis $A^C =
\Omega \\ A$ Teil der Ergebnisalgebra}
    \item{Für jedes Ereignispaar $A,B$ ist sowohl das Ereignis "`A und B treten
ein"' als auch "`A oder B treten ein"' Teil der Ergebnisalgebra}
  \end{itemize}

Die Ergebnisalgebra ist also unter den Operationen Komplementbildung, $\cup$ und
$\cap$ abgeschlossen.
\end{definition}

\begin{example}{Ereignisalgebra}{ealg}
Durch die Definition der Ereignisalgebra muss nicht immer die gesamte Potenzmenge
eines Grundraums betrachtet werden. Im Beispiel des Würfelwurfs ($\Omega =
\{1,2,3,4,5,6\}$) bildet auch
\[\{\emptyset, \Omega, \{1,3,5\}, \{2,4,6\}\}\]

eine Ereignisalgebra. Da die paarweise Verknüpfung mit $\cup$ und $\cap$ bzw. das
Komplement eines Ereignisses immer auch in der Algebra vorhanden sind und ein
sicheres und unmögliches Ereignis vorhanden sind, lässt sich bereits sinnvolle
Wahrscheinlichkeitsbetrachtungen anstellen.

Die Ereignisse $A_1, A_2, \Omega, \emptyset$ aus Beispiel \ref{bsp:ereignis}
bilden keine Ereignisalgebra, da (unter anderem) $A_1 \cup A_2 = \{3\}$ ein
neues Ereignis ergibt.
\end{example}


\section{Wahrscheinlichkeit}

Die \emph{Wahrscheinlichkeit} eines Ereignisses lässt sich auch statistisch als
relative Häufigkeit des Auftretens im Verhältnis zur Anzahl der durchgeführten
Versuche des Zufallsexperiments beschreiben. Um die Wahrscheinlichkeit auf diese
Weise zuverlässig bestimmen zu können, müssen sehr viele Zufallsversuche
durchgeführt werden. Etwas "`mathematischer"' ist folgende (axiomatische)
Definition:

\begin{definition}{Wahrscheinlichkeit}{whkt}
Sei $\mathcal{A}$ eine \link{def:ealg}{Ereignisalgebra} auf einem
\link{def:grundraum}{Grundraum} $\Omega$. Eine Abbildung
\[ P: \mathcal{A} \to 0,1 \]

heißt \defw{Wahrscheinlichkeit}, wenn sie folgende Bedingungen erfüllt:
\[\forall A \subseteq \Omega: 0 \leq P(A) \leq 1\]
\[P(\Omega) = 1\]
\[\forall A_1, A_2, ... A_n \subseteq \Omega\ paarweise\ disjunkt:
P(\bigcup_{i=1}^{n} A_i) = \sum_{i=1}^{n}P(A_i))\]
\end{definition}

Wahrscheinlichkeiten von Ereignissen sind also immer Werte im Bereich von 0
(unmöglich) und 1 (sicher). Zusätzlich kann man die Wahrscheinlichkeit eines
nicht-elementaren Ereignisses durch Zerlegung in disjunkte Teilereignisse
berechnen.


\section{Unabhängigkeit und bedingte Wahrscheinlichkeit}

Die bedingte Wahrscheinlichkeit ist immer dann relevant, wenn die Beziehung
zweier Ereignisse untersucht wird. Der Begriff beschreibt die Wahrscheinlichkeit
eines Ereignisses, wenn man davon ausgeht, dass ein anderes Ereignis (die
\emph{Bedingung}) bereits eingetreten ist. Ändern sich die Wahrscheinlichkeiten
durch die Bedingung nicht, sind die Ereignisse unabhängig.

\begin{definition}{Bedingte Wahrscheinlichkeit}{bedw}
Seien $A, B$ \link{def:ereignis}{Ereignisse} und $P(B)>0$. Dann heißt
\[P(A|B) := \frac{P(A\cap B)}{P(B)}\]
\defw{bedingte Wahrscheinlichkeit} von $A$ unter der Bedingung $B$.
Eine Schreibweise der bedingten Wahrscheinlichkeit $P(A|B)$ ist $P_B(A)$.
\end{definition}

\begin{theorem}{Satz von Bayes}{bayes}
Seien $A,B$ \link{def:ereignis}{Ereignisse}. Es gilt:
\[P(A|B) = P(B|A) \cdot\frac{P(A)}{P(B)}\]
Der Satz folgt aus der Definition der bedingten Wahrscheinlichkeit und $P(A\cap
B) = P(B\cap A)$.
\end{theorem}
Durch den Satz von Bayes können also zum Beispiel bedingte Wahrscheinlichkeiten
"`umgedreht"' werden.

\begin{definition}{Unabhängigkeit von Ereignissen}{unabh}
Zwei \link{def:ereignis}{Ereignisse} $A, B$ heißen \defw{unabhängig}, wenn gilt:
\[P(A|B) = P(A) \iff P(B|A) = P(B)\]
Ereignisse $A_1,...,A_n$ heißen unabhängig, wenn gilt:
\[P(A_1\cap ...\cap A_n) = P(A_1)\cdot ...\cdot P(A_n)\]
\end{definition}

Sind Ereignisse unabhängig, kann man die Wahrscheinlichkeit des gemeinsamen
Auftretens durch Multiplikation ermitteln:
\[P(A\cap B) = P(A|B)\cdot P(B) = P(A)\cdot P(B)\]
Sind die Ereignisse nicht unabhängig, ist $P(A|B) \ne P(A)$, sodass die
Einzelwahrscheinlichkeiten nicht einfach multipliziert werden dürfen.

\begin{theorem}{Multiplikationssatz}{multiplikation}
Seien $A_1, ...,A_n$ zufällige Ereignisse mit $P(A_i) > 0)$. Dann gilt:
\[P(A_1\cap A_2\cap ...\cap A_n = P(A_1)\cdot P(A_2|A_1) \cdot ...\cdot
P(A_n|A_1\cap ...\cap A_{n-1})\]
\end{theorem}


\subsection{Zufallsvariablen}

\begin{definition}{Zufallsvariable}{zvar}
Eine Funktion $X: \Omega \to \R$ wird als \defw{Zufallsvariable}
bezeichnet. Die Zufallsvariable ordnet jedem Ereignis einer
\link{def:ealg}{Ereignisalgebra} eine reelle Zahl zu.

Der Wertebereich der Zufallszahl wird als \altname{Zustandsraum} $S = X(\Omega)$
bezeichnet.

Ist $S$ endlich oder abzählbar unendlich, wird die Zufallsvariable als
\defw{diskret}, ist $S$ überabzählbar unendlich als \defw{stetig} bezeichnet.
\end{definition}

\begin{definition}{Unabhängigkeit von Zufallsvariablen}{zvar-unabh}
Seien $X: \Omega \to \R$, $Y: \Omega \to \R$ \link{def:zvar}{Zufallsvariablen}. $X,
Y$ heißen \defw{unabhängig}, wenn für $a,b,c,d \in \R$, $a\le b, c\le d$ gilt:
\[P(a < X\le b,c<Y\le d) = P(a<X\le b)\cdot P(c<Y\le d)\]
\end{definition}
