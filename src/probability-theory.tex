\chapter{Grundlagen der Wahrscheinlichkeitstheorie}

Die Wahrscheinlichkeitstheorie ist ein Teilgebiet der Mathematik, dass sich mit
der Formalisierung, Modellierung und Untersuchung von zufälligen Vorgängen
beschäftigt (\href{https://de.wikipedia.org/wiki/Wahrscheinlichkeitstheorie}
{Wikipedia}).

\section{Zufallsversuch, Ereignis, Zufallsvariable}

\begin{definition}[Zufallsversuch]
Ein Versuch oder Vorgang, der unter genau festgelegten Versuchsbedingungen
durchgeführt einen zufälligen Ausgang besitzt, wird als \defw{Zufallsversuch}
(auch: \altname{Zufallsexperiment} oder \altname{Wahrscheinlichkeitsexperiment})
bezeichnet.
\end{definition}

\begin{definition}[Ergebnismenge]
Alle möglichen Ausgänge eines Zufallsversuchs bilden die \defw{Ergebnismenge}
$\Omega$ (auch: \altname{Grundraum}). Die Elemente der Ergebnismenge werden auch als
\altname{Elementarereignisse} bezeichnet.
\end{definition}

\begin{definition}[Ereignis]
Eine Teilmenge $A$ von $\Omega$ wird als Ereignis bezeichnet. Dabei bezeichnet
$A = \Omega$ das sichere Ereignis, dass immer eintritt und $A = \emptyset$ das
unmögliche Ereignis.
\end{definition}

\defw{Beispiel:} Würfelwurf mit $\Omega = \{1,2,3,4,5,6\}$ mit Ereignissen
"`ungerade Augenzahl"' $A_1 = \{1,3,5\}$ und "`2 oder 3"' $A_2 = \{2,3\}$.

\begin{definition}[Zufallsvariable]
Eine Funktion $X: \Omega \to R$ wird als \defw{Zufallsvariable}
bezeichnet. Die Zufallsvariable ordnet jedem Elementarereignis eine reelle Zahl
zu.

Der Wertebereich der Zufallszahl wird als \altname{Zustandsraum} $S = X(\Omega)$
bezeichnet.
\end{definition}

\section{Wahrscheinlichkeit}

Die \emph{Wahrscheinlichkeit} eines Ereignisses lässt sich auch statistisch als
relative Häufigkeit des Auftretens im Verhältnis zur Anzahl der durchgeführten
Versuche des Zufallsexperiments beschreiben. Um die Wahrscheinlichkeit auf diese
Weise zuverlässig bestimmen zu können, müssen sehr viele Zufallsversuche
durchgeführt werden. Etwas "`mathematischer"' ist folgende Definition:

\begin{definition}[Wahrscheinlichkeit]
Sei $\mathcal{A}$ eine Menge von Ereignissen auf einem Grundraum $\Omega$. Eine
Abbildung

\[ P: \mathcal{A} \to 0,1 \]

heißt \defw{Wahrscheinlichkeit}, wenn sie folgende Bedingungen erfüllt:

\[\forall A \subseteq \Omega: 0 \leq P(A) \leq 1\]
\[P(\Omega) = 1\]
\[\forall A_1, A_2, ... A_n \subseteq \Omega\ paarweise\ disjunkt:
P(\bigcup_{i=1}^{n} A_i) = \sum_{i=1}^{n}P(A_i))\]
\end{definition}

Wahrscheinlichkeiten von Ereignissen sind also immer Werte im Bereich von 0
(unmöglich) und 1 (sicher). Zusätzlich kann man die Wahrscheinlichkeit eines
nicht-elementaren Ereignisses durch Zerlegung in disjunkte Teilereignisse
berechnen.
